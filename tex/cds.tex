\documentclass[9pt,twocolumn]{article}

\usepackage[margin=0.8in,bottom=1.25in,columnsep=.4in]{geometry}
\usepackage{amsmath}
\usepackage{amssymb}
\usepackage{listings}
\usepackage{color}

\DeclareMathOperator*{\argmin}{arg\,min}
\DeclareMathOperator*{\argmax}{arg\,max}

\definecolor{dkgreen}{rgb}{0,0.6,0}
\definecolor{gray}{rgb}{0.5,0.5,0.5}
\definecolor{mauve}{rgb}{0.58,0,0.82}
\lstset{frame=tb,
  language=Bash,
  aboveskip=3mm,
  belowskip=3mm,
  showstringspaces=false,
  columns=flexible,
  basicstyle={\small\ttfamily},
  numbers=none,
  numberstyle=\tiny\color{gray},
  keywordstyle=\color{blue},
  commentstyle=\color{dkgreen},
  stringstyle=\color{mauve},
  breaklines=true,
  breakatwhitespace=true,
  tabsize=3
}

\title{
	Computational Data Science
}

\author{Joel Huang}

\date{\today}

\begin{document}
\maketitle

\section{Entity Relationship Diagrams \& SQL}



\section{Lab: Data Handling in Unix}

\subsection*{Previewing data}
Raw data can be fairly large. We might want to examine the dataset's structure without opening it in a special program or text editor, which could take a long time and large amounts of memory.

\subsubsection*{Preview all - \lstinline{cat}}
Short for con\textbf{cat}enate. Sequentially reads file(s) and writes them to \lstinline{stdout}. If redirection \lstinline{>} is used, then output is written to the specified file. \lstinline{>} is used to write to a file and \lstinline{>>} is used to append to a file.
\begin{lstlisting}
cat file1.txt
cat file1.txt file2.txt > newcombinedfile.txt
cat >newfile.txt
cat -n file1.txt file2.txt > newnumberedfile.txt
cat file1.txt >> file2.txt
cat file1.txt file2.txt file3.txt | sort > test4
\end{lstlisting}

\subsubsection*{Preview some - \lstinline{head, tail}}
Use \lstinline{head} or \lstinline{tail} to preview the head or tail of the data. Remember to supply the flags:
\begin{lstlisting}
-n  Number of lines
-B  Display number of lines before
-A  Display number of lines after
\end{lstlisting}

\subsection*{Searching - \lstinline{grep}}
\begin{lstlisting}
-E  Use extended regular expression syntax
-o  Output a matching segment of each line only
-n  Print the line number of each matched line
-C  Show a number of context lines too
\end{lstlisting}


\begin{lstlisting}
.   Matches any character.
*   Matches zero or more instances of the preceding character.
+   Matches one or more instances of the preceding character.
[]  Matches any of the characters within the brackets.
()  Creates a sub-expression that can be combined to make more complicated expressions.
|   OR operator; (www|ftp) matches either 'www' or 'ftp'.
^   Matches the beginning of a line.
$   Matches the end of the line.
\   Escapes the following character. Since . matches any character, to match a literal period you would need to use \..
\end{lstlisting}
%\subsection*{Replacing}
%\subsection*{Previewing}




\section{Big Data, Hadoop, \& MapReduce}
\section{Lab: MapReduce \& Hadoop}

\subsection*{Big Idea: Mappers \& Reducers}
\section{Counting Relative Frequencies}
\subsection*{Big Idea: Approximate probabilities by counting occurrences in the data}
\subsubsection*{k-Nearest Neighbours}
A way to measure similarity between different data points, through determining similarity values or distances. Classification with k-NN is straightforward, but sensitive to the value of $k$, potentially computationally expensive (but can be sped up using kd-tree), and takes memory to store all data points.

\subsubsection*{Decision trees \& Random Forests}
Each feature represents an opportunity for branching. Decision trees are inexpensive and explainable, but prone overfitting withouot pruning
To prevent overfitting, use Random forests-ensemble approach that aggregates decision trees's outputs via a majority voting system

\subsubsection*{Naive Bayes for classification}
\begin{lstlisting}
Assumption: Conditional independence
\end{lstlisting}
\begin{equation}
	P(x_1, x_2, \mathellipsis, x_n\,|\,Y) = P(x_1\,|\,Y)\cdot P(x_2\,|\,Y)\cdot\mathellipsis\
\end{equation}
We are trying to find the most likely class given an observation
Maximum
Most likely class Y given feature set X
\begin{equation}
	y_{MAP} = \argmax\,P(Y\,|\, x_1, x_2, \mathellipsis, x_m)
\end{equation}
\begin{equation}
	= \argmax\,\frac{P(X\,|\,Y)\,P(Y)}{P(X)}
\end{equation}
In equation (2), we're comparing the $\argmax$ of $P(Y=0\,|\,X)$ and $P(Y=1\,|\,X)$. Therefore we can cancel the constant denominator $P(X)$:
\begin{equation}
	= \argmax\,P(X\,|\,Y)\,P(Y)
\end{equation}
Next, we need to find $P(X\,|\,Y)$ and $P(Y)$.

\paragraph*{Count the prior probability $P(Y)$}
Class labels are $Y\in\{0,1\}$. Therefore, for a dataset with $n$ labels,
\begin{equation}
	P(Y=0) = \frac{n_{Y=0}}{n}
\end{equation}
\begin{equation}
	P(Y=1) = \frac{n_{Y=1}}{n}
\end{equation}
In other words, there are $n_{Y=1}$ out of $n$ occurrences of label $Y$ having value 1.
\paragraph*{Estimate $P(X\,|\,Y=0)$ and $P(X\,|\,Y=1)$}
Since we have assumed conditional independence of $X$ given classes $Y$ (1),
\begin{equation}
	P(x_i\,|\,Y=0) = \frac{|x_{i,Y=0}|}{n_{Y=0}}
\end{equation}
We are counting the probability (occurrences in the dataset) of feature $x_i$ given $Y=0$. Find all the rows where $Y=0$ and count the occurrences of feature $x_i$. Repeat this for $Y=1$.

\paragraph*{Final comparison ($\argmax$)}
Now we have pairs $P(Y=0)$, $P(X\,|\,Y=0)$, and $P(Y=1)$, $P(X\,|\,Y=1)$. Multiply the pairs as in equation (4)	and take the $max$ of the two.

\subsubsection*{Naive Bayes for word counting}
\begin{lstlisting}
Assumptions: Conditional independence, irrelevance of word order
\end{lstlisting}
We want to predict a class for a given sentence, for example, ``good'' for the sentence ``love the burgers here, delicious and filling''. We can estimate the class $c$ using Naive Bayes,
\begin{equation}
	c = \argmax	_{c_i\in C} P(c_i) \prod_{w_j\in W} P(w_j\,|\,c_i)
\end{equation}
For a set of documents $D$, words $W$, and classes $C$, the probability of class $c_i$ is simply the number of occurrences of $c_i$ in document $D$.

\begin{equation}
	P(c_i) = \frac{|c_i|}{|D|}
\end{equation}
The probability of word $w_1$ given class $c_i$ is also simply the number of occurrences of word $w_1$ in sentences with class $c_i$, divided by the probability of class $c_i$.
\begin{equation}
\begin{split}
	P(w_1\,|\,c_i) = \frac{P(w_1,c_i)}{P(c_i)}\\
	= \frac{count(w_1,c_i)}{\sum_{w_j\in W}count(w_j,c_i)}
\end{split}
\end{equation}

\paragraph*{Example: Restaurant ratings}
\begin{center}
\begin{tabular}{ |c|c|c| } 
\hline
ID & Sentence & Class \\
\hline
1& The burger is \textbf{tasteless} & \\
  & and \textbf{slow} service        & Bad\\
2& \textbf{slow} serving time and & \\
 & everything is \textbf{horrible} & Bad\\
3& Restaurant is near MRT, & \\
 & serves \textbf{delicious} burgers & Good\\
4& \textbf{love} the burger here, & \\
 & \textbf{delicious} and filling & Good\\
\hline
5& \textbf{love} this place, \textbf{delicious} & \\
 & burgers but \textbf{slow} service & ?\\
\hline
\end{tabular}
\end{center}
Where classes $C = \{Bad\,,\,Good\}$, and words $W = \{tasteless, slow, horrible, delicious, love\}$.
\begin{equation}
	P(c) = 
	\begin{cases}
      \frac{2}{4}, & c = Bad \\
      \frac{2}{4}, & c = Good
    \end{cases}
\end{equation}

\begin{equation}
\begin{split}
P(love\,|\,Good) = \frac{count(love\,,\,Good)}{\sum_{w_j\in W}count(w_j\,,\,Good)} = \frac{1}{3}\\
P(delicious\,|\,Good) = \frac{count(delicious\,,\,Good)}{\sum_{w_j\in W}count(w_j\,,\,Good)} = \frac{2}{3}\\
\end{split}
\end{equation}

\textbf{* TODO: CHECK THIS}
\begin{equation}
\begin{split}
P(Good\,|\,{love, delicious}) = P(Good)\prod_{w_j\in W}P(w_j\,|\,Good)\\
= P(Good)\cdot P(love\,|\,Good)\cdot P(delicious\,|\,Good)\\
= \frac{2}{4}\times\frac{1}{3}\times\frac{2}{3} = \frac{1}{9}
\end{split}
\end{equation}

\begin{equation}
\begin{split}
P(tasteless\,|\,Bad) = \frac{count(tasteless\,,\,Bad)}{\sum_{w_j\in W}count(w_j\,,\,Bad)} = \frac{1}{4}\\
P(slow\,|\,Bad) = \frac{count(slow\,,\,Bad)}{\sum_{w_j\in W}count(w_j\,,\,Bad)} = \frac{2}{4}\\
P(horrible\,|\,Bad) = \frac{count(horrible\,,\,Bad)}{\sum_{w_j\in W}count(w_j\,,\,Bad)} = \frac{1}{4}
\end{split}
\end{equation}

\textbf{* TODO: INCOMPLETE}
\begin{equation}
\begin{split}
P(Bad\,|\,{love, delicious}) = P(Bad)\prod_{w_j\in W}P(w_j\,|\,Bad)\\
= P(Bad)\cdot P(love\,|\,Bad)\cdot P(delicious\,|\,Bad)\\
= \frac{2}{4}\times\frac{1}{3}\times\frac{2}{3} = \frac{1}{9}
\end{split}
\end{equation}

% \section{Evaluation Methodologies \& Metrics}
% \section{Lab: Naive Bayes}
% \section{Data Visualization}
% \section{Lab: Visualization}
% \section{Feature Vectors}
% \section{Time Series Analysis}
% \section{Lab: Time Series Mining}
% \section{Clustering \& Community Detection}
% \section{Community Detection}
% \section{Deep Learning}
% \section{Lab: Multi-Layer Perceptrons}
% \section{Sentiment Analysis}
% \section{Lab: Word2Vec}
% \section{Convolutional Neural Networks}
% \section{Lab: Convolutional Neural Networks}
% \section{Temporal Sequences \& Memory Models}
% \section{Lab: RNN \& LSTM}



\end{document}